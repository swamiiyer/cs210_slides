\documentclass[8pt,a4paper,compress]{beamer}

\usepackage{/home/siyer/lib/slides}

\title{Mathematical Preliminaries}
\date{}

\newlength{\myMheight}
\settoheight{\myMheight}{M}

\begin{document}
\begin{frame}
\vfill
\titlepage
\end{frame}

\section{Numbers}
\begin{frame}[fragile]
\pause\transdissolve

An integer is a number that can be written without a fractional component

\pause\transdissolve\bigskip

Examples of integers: -128, 0, 42

\pause\transdissolve\bigskip

An integer is even if it is divisible by 2, and odd otherwise

\pause\transdissolve\bigskip

The set of all integers is denoted as $\mathbb{Z}$

\pause\transdissolve\bigskip

The set $\mathbb{Z}$ includes zero (0), the natural numbers $\mathbb{N}$ ($1, 2, 3, \dots$), and their additive inverses ($-1, -2, -3, \dots$)
\end{frame}

\begin{frame}[fragile]
\pause\transdissolve

A rational number is a number that can be expressed as the quotient or fraction $p/q$ of two integers, a numerator $p$ and a non-zero denominator $q$

\pause\transdissolve\bigskip

Examples of rational numbers: 42, 2.5, 1.333...

\pause\transdissolve\bigskip

The set of all rational numbers is denoted as $\mathbb{Q}$
\end{frame}

\begin{frame}[fragile]
\pause\transdissolve

A real number is a value of a continuous quantity such as distance or time

\pause\transdissolve\bigskip

Examples of real numbers: 1.333..., $\sqrt{2}$ (1.41421...), $\pi$ (3.14159...), $e$ (2.71828...)

\pause\transdissolve\bigskip

The set of all real numbers is denoted as $\mathbb{R}$

\pause\transdissolve\bigskip

Hierarchy of number systems ($\subset$ means proper subset): $\mathbb{N} \subset \mathbb{Z} \subset \mathbb{Q} \subset \mathbb{R}$

\pause\transdissolve\bigskip

A real number that is not rational is called an irrational number

\pause\transdissolve\bigskip

Examples of irrational numbers: $\sqrt{2}$ (1.41421...), $\pi$ (3.14159...), $e$ (2.71828...)
\end{frame}

\begin{frame}[fragile]
\pause\transdissolve

A prime number is a natural number greater than 1 that cannot be formed by multiplying two smaller natural numbers

\pause\transdissolve\bigskip

Examples of prime numbers: 5, 43, 97

\pause\transdissolve\bigskip

There are infinitely many primes

\pause\transdissolve\bigskip

A natural number greater than 1 that is not prime is called a composite number

\pause\transdissolve\bigskip

Examples of composite numbers: 8, 46, 99

\pause\transdissolve\bigskip

Prime factorization is the decomposition of a composite number into a product of primes

\pause\transdissolve\bigskip

For example, the prime factorization of 18 is $2 \times 3 \times 3$
\end{frame}

\section{Operations on Numbers}
\begin{frame}[fragile]
\pause\transdissolve

Addition ($x + y$)

\pause\transdissolve\bigskip

Subtraction ($x - y$)

\pause\transdissolve\bigskip

Multiplication ($x \times y$)

\pause\transdissolve\bigskip

Division ($x \div y$)

\pause\transdissolve\bigskip

Exponentiation ($x^y$)

\pause\transdissolve\bigskip

Order of operations: (P)arenthesis, (E)xponents, (M)ultiplication, (D)ivision, (A)ddition, (S)ubtraction

\pause\transdissolve\bigskip

For example, $(5 - 3) \times 2^3 + 24 \div 4 = 22$

\pause\transdissolve\bigskip

If $x$ and $d$ ($\neq 0$) are integers, then there exist unique integers $q$ (quotient) and $r$ (remainder), such that $x = q \times d + r$ and $0 \leq r \leq |d|$

\pause\transdissolve\bigskip

For example, $43 = 8 \times 5 + 3$
\end{frame}

\begin{frame}[fragile]
\pause\transdissolve

Laws of exponents
\visible<2->{
\tcbox[enhanced,drop shadow southwest,sharp corners,size=fbox,colback=white]{
\begin{tabular}{ll}
\textbf{Law} & \textbf{Example} \\ \hline \\
$x^1=x$ & $6^1=6$ \\
$x^0=1$ & $7^0=1$ \\
$x^{-n}=1/x^n$ & $x^{-3}=1/x^3$ \\
$x^mx^n=x^{m+n}$ & $x^2x^3=x^5$ \\ 
$x^m/x^n=x^{m-n}$ & $x^6/x^2=x^4$ \\ 
$(x^m)^n=x^{mn}$ & $(x^2)^3=x^6$ \\
$(xy)^n=x^ny^n$ & $(xy)^3=x^3y^3$ \\
$(x/y)^n=x^n/y^n$ & $(x/y)^2=x^2/y^2$ \\
$x^{m/n}=\sqrt[n]{x^m}=(\sqrt[n]{x})^m$ & $x^{2/3}=\sqrt[3]{x^2}=(\sqrt[3]{x})^2$ \\
\end{tabular}
}
}
\end{frame}

\section{Functions}
\begin{frame}[fragile]
\pause\transdissolve

A function $y = f(x)$ of a single variable can be thought of as a machine that takes a single input $x$ and produces an output $y$

\begin{center}
\visible<2->{
\begin{tikzpicture}
\footnotesize
\begin{scope}[->,thin,
	   node distance=0.5cm,
  	   block1/.style={rectangle,draw,align=center},
	   block2/.style={rectangle,align=center}]
\node [block2] (1) {$x$};
\node [block1] (2) [right=of 1] {$f(x)$};
\node [block2] (3) [right=of 2] {$y$};
\path (1) edge node [above] {} (2);
\path (2) edge node [above] {} (3);
\end{scope}
\end{tikzpicture}
}
\end{center}

\pause\transdissolve\bigskip

For example, consider the function $f(x) = x^2$

\bigskip

\begin{overprint}
\onslide<3>
\begin{center}
\begin{tikzpicture}
\footnotesize
\begin{scope}[->,thin,
	   node distance=0.5cm,
  	   block1/.style={rectangle,draw,align=center},
	   block2/.style={rectangle,align=center}]
\node [block2] (1) {$x$};
\node [block1] (2) [right=of 1] {$x^2$};
\node [block2] (3) [right=of 2] {$y$};
\path (1) edge node [above] {} (2);
\path (2) edge node [above] {} (3);
\end{scope}
\end{tikzpicture}
\end{center}

\onslide<4>
\begin{center}
\begin{tikzpicture}
\footnotesize
\begin{scope}[->,thin,
	   node distance=0.5cm,
  	   block1/.style={rectangle,draw,align=center},
	   block2/.style={rectangle,align=center}]
\node [block2] (1) {$-5$};
\node [block1] (2) [right=of 1] {$x^2$};
\node [block2] (3) [right=of 2] {$25$};
\path (1) edge node [above] {} (2);
\path (2) edge node [above] {} (3);
\end{scope}
\end{tikzpicture}
\end{center}

\onslide<5>
\begin{center}
\begin{tikzpicture}
\footnotesize
\begin{scope}[->,thin,
	   node distance=0.5cm,
  	   block1/.style={rectangle,draw,align=center},
	   block2/.style={rectangle,align=center}]
\node [block2] (1) {$11$};
\node [block1] (2) [right=of 1] {$x^2$};
\node [block2] (3) [right=of 2] {$121$};
\path (1) edge node [above] {} (2);
\path (2) edge node [above] {} (3);
\end{scope}
\end{tikzpicture}
\end{center}
\end{overprint}
\end{frame}

\begin{frame}[fragile]
\pause\transdissolve

As another example, consider the function $f(F) = \frac{5}{9}(F - 32)$ that converts temperature $F$ from Fahrenheit to Celsius

\bigskip

\begin{overprint}
\onslide<3>
\begin{center}
\begin{tikzpicture}
\footnotesize
\begin{scope}[->,thin,
	   node distance=0.5cm,
  	   block1/.style={rectangle,draw,align=center},
	   block2/.style={rectangle,align=center}]
\node [block2] (1) {$F$};
\node [block1] (2) [right=of 1] {$\frac{5}{9}(F - 32)$};
\node [block2] (3) [right=of 2] {$C$};
\path (1) edge node [above] {} (2);
\path (2) edge node [above] {} (3);
\end{scope}
\end{tikzpicture}
\end{center}

\onslide<4>
\begin{center}
\begin{tikzpicture}
\footnotesize
\begin{scope}[->,thin,
	   node distance=0.5cm,
  	   block1/.style={rectangle,draw,align=center},
	   block2/.style={rectangle,align=center}]
\node [block2] (1) {$32$};
\node [block1] (2) [right=of 1] {$\frac{5}{9}(F - 32)$};
\node [block2] (3) [right=of 2] {$0$};
\path (1) edge node [above] {} (2);
\path (2) edge node [above] {} (3);
\end{scope}
\end{tikzpicture}
\end{center}

\onslide<5>
\begin{center}
\begin{tikzpicture}
\footnotesize
\begin{scope}[->,thin,
	   node distance=0.5cm,
  	   block1/.style={rectangle,draw,align=center},
	   block2/.style={rectangle,align=center}]
\node [block2] (1) {$212$};
\node [block1] (2) [right=of 1] {$\frac{5}{9}(F - 32)$};
\node [block2] (3) [right=of 2] {$100$};
\path (1) edge node [above] {} (2);
\path (2) edge node [above] {} (3);
\end{scope}
\end{tikzpicture}
\end{center}
\end{overprint}
\end{frame}

\begin{frame}[fragile]
\pause\transdissolve

The root(s) of a function are the values of $x$ that satisfy the equation $f(x) = 0$

\pause\transdissolve\bigskip

For example, the two roots of the function $f(x) = x^2 - 4$ are $x = \pm 2$
\end{frame}


\begin{frame}[fragile]
\pause\transdissolve

A polynomial of degree $n$ is a function of the form $$f(x) = a_nx^n + a_{n-1}x^{n-1} + \cdots + a_2x^2 + a_1x + a_0,$$ where $a_0, \dots, a_n$ are constants and $x$ is variable

\begin{center}
\visible<2->{
\begin{tikzpicture}
\footnotesize
\begin{scope}[->,thin,
	   node distance=0.5cm,
  	   block1/.style={rectangle,draw,align=center},
	   block2/.style={rectangle,align=center}]
\node [block2] (1) {$x$};
\node [block1] (2) [right=of 1] {$a_nx^n + a_{n-1}x^{n-1} + \cdots + a_2x^2 + a_1x + a_0$};
\node [block2] (3) [right=of 2] {$y$};
\path (1) edge node [above] {} (2);
\path (2) edge node [above] {} (3);
\end{scope}
\end{tikzpicture}
}
\end{center}

\pause\transdissolve\bigskip

A polynomial of degree 0 is called a constant function

\visible<3->{
\begin{center}
\begin{tikzpicture}
\footnotesize
\begin{scope}[->,thin,
	   node distance=0.5cm,
  	   block1/.style={rectangle,draw,align=center},
	   block2/.style={rectangle,align=center}]
\node [block2] (1) {$x$};
\node [block1] (2) [right=of 1] {$a_0$};
\node [block2] (3) [right=of 2] {$a_0$};
\path (1) edge node [above] {} (2);
\path (2) edge node [above] {} (3);
\end{scope}
\end{tikzpicture}
}
\end{center}
\end{frame}

\begin{frame}[fragile]
\pause\transdissolve

A polynomial of degree 1 is called a linear function

\visible<2->{
\begin{center}
\begin{tikzpicture}
\footnotesize
\begin{scope}[->,thin,
	   node distance=0.5cm,
  	   block1/.style={rectangle,draw,align=center},
	   block2/.style={rectangle,align=center}]
\node [block2] (1) {$x$};
\node [block1] (2) [right=of 1] {$a_1x + a_0$};
\node [block2] (3) [right=of 2] {$y$};
\path (1) edge node [above] {} (2);
\path (2) edge node [above] {} (3);
\end{scope}
\end{tikzpicture}
}
\end{center}

\pause\transdissolve\bigskip

The linear function is usually written in the form $y = mx + c$

\pause\transdissolve\bigskip

A polynomial of degree 2 is called a quadratic function

\visible<3->{
\begin{center}
\begin{tikzpicture}
\footnotesize
\begin{scope}[->,thin,
	   node distance=0.5cm,
  	   block1/.style={rectangle,draw,align=center},
	   block2/.style={rectangle,align=center}]
\node [block2] (1) {$x$};
\node [block1] (2) [right=of 1] {$a_2x^2 + a_1x + a_0$};
\node [block2] (3) [right=of 2] {$y$};
\path (1) edge node [above] {} (2);
\path (2) edge node [above] {} (3);
\end{scope}
\end{tikzpicture}
}
\end{center}

\pause\transdissolve\bigskip

The quadratic function is usually written in the form $y = ax^2 + bx + c$

\pause\transdissolve\bigskip

The roots of the quadratic equation $ax^2 + bx + c = 0$ are $$x = \frac{-b \pm \sqrt{b^2-4ac}}{2a}$$

\end{frame}

\begin{frame}[fragile]
\pause\transdissolve

An exponentiation function is of the form $f(x) = b^x$, where $b$ is constant and $x$ is variable

\visible<2->{
\begin{center}
\begin{tikzpicture}
\footnotesize
\begin{scope}[->,thin,
	   node distance=0.5cm,
  	   block1/.style={rectangle,draw,align=center},
	   block2/.style={rectangle,align=center}]
\node [block2] (1) {$x$};
\node [block1] (2) [right=of 1] {$b^x$};
\node [block2] (3) [right=of 2] {$y$};
\path (1) edge node [above] {} (2);
\path (2) edge node [above] {} (3);
\end{scope}
\end{tikzpicture}
}
\end{center}

\pause\transdissolve\bigskip

For example, consider the function $f(x) = 3^x$

\bigskip

\begin{overprint}
\onslide<3>
\begin{center}
\begin{tikzpicture}
\footnotesize
\begin{scope}[->,thin,
	   node distance=0.5cm,
  	   block1/.style={rectangle,draw,align=center},
	   block2/.style={rectangle,align=center}]
\node [block2] (1) {$x$};
\node [block1] (2) [right=of 1] {$3^x$};
\node [block2] (3) [right=of 2] {$y$};
\path (1) edge node [above] {} (2);
\path (2) edge node [above] {} (3);
\end{scope}
\end{tikzpicture}
\end{center}

\onslide<4>
\begin{center}
\begin{tikzpicture}
\footnotesize
\begin{scope}[->,thin,
	   node distance=0.5cm,
  	   block1/.style={rectangle,draw,align=center},
	   block2/.style={rectangle,align=center}]
\node [block2] (1) {$-1$};
\node [block1] (2) [right=of 1] {$3^x$};
\node [block2] (3) [right=of 2] {$0.333...$};
\path (1) edge node [above] {} (2);
\path (2) edge node [above] {} (3);
\end{scope}
\end{tikzpicture}
\end{center}

\onslide<5>
\begin{center}
\begin{tikzpicture}
\footnotesize
\begin{scope}[->,thin,
	   node distance=0.5cm,
  	   block1/.style={rectangle,draw,align=center},
	   block2/.style={rectangle,align=center}]
\node [block2] (1) {$4$};
\node [block1] (2) [right=of 1] {$3^x$};
\node [block2] (3) [right=of 2] {$81$};
\path (1) edge node [above] {} (2);
\path (2) edge node [above] {} (3);
\end{scope}
\end{tikzpicture}
\end{center}
\end{overprint}
\end{frame}

\begin{frame}[fragile]
\pause\transdissolve

The logarithm function $f(x) = \log_b(x)$ computes the power to which $b$ (constant) must be raised to obtain $x$ (variable)

\visible<2->{
\begin{center}
\begin{tikzpicture}
\footnotesize
\begin{scope}[->,thin,
	   node distance=0.5cm,
  	   block1/.style={rectangle,draw,align=center},
	   block2/.style={rectangle,align=center}]
\node [block2] (1) {$x$};
\node [block1] (2) [right=of 1] {$\log_b(x)$};
\node [block2] (3) [right=of 2] {$y$};
\path (1) edge node [above] {} (2);
\path (2) edge node [above] {} (3);
\end{scope}
\end{tikzpicture}
}
\end{center}

\pause\transdissolve\bigskip

The logarithm function is the inverse of the exponentiation function

\pause\transdissolve\bigskip

For example, consider the function $f(x) = \log_2(x)$

\bigskip

\begin{overprint}
\onslide<4>
\begin{center}
\begin{tikzpicture}
\footnotesize
\begin{scope}[->,thin,
	   node distance=0.5cm,
  	   block1/.style={rectangle,draw,align=center},
	   block2/.style={rectangle,align=center}]
\node [block2] (1) {$x$};
\node [block1] (2) [right=of 1] {$\log_2(x)$};
\node [block2] (3) [right=of 2] {$y$};
\path (1) edge node [above] {} (2);
\path (2) edge node [above] {} (3);
\end{scope}
\end{tikzpicture}
\end{center}

\onslide<5>
\begin{center}
\begin{tikzpicture}
\footnotesize
\begin{scope}[->,thin,
	   node distance=0.5cm,
  	   block1/.style={rectangle,draw,align=center},
	   block2/.style={rectangle,align=center}]
\node [block2] (1) {$1$};
\node [block1] (2) [right=of 1] {$\log_2(x)$};
\node [block2] (3) [right=of 2] {$0$};
\path (1) edge node [above] {} (2);
\path (2) edge node [above] {} (3);
\end{scope}
\end{tikzpicture}
\end{center}

\onslide<6>
\begin{center}
\begin{tikzpicture}
\footnotesize
\begin{scope}[->,thin,
	   node distance=0.5cm,
  	   block1/.style={rectangle,draw,align=center},
	   block2/.style={rectangle,align=center}]
\node [block2] (1) {$64$};
\node [block1] (2) [right=of 1] {$\log_2(x)$};
\node [block2] (3) [right=of 2] {$6$};
\path (1) edge node [above] {} (2);
\path (2) edge node [above] {} (3);
\end{scope}
\end{tikzpicture}
\end{center}
\end{overprint}
\end{frame}

\begin{frame}[fragile]
\pause\transdissolve

Basic properties of logarithms
\begin{enumerate}
\pause\transdissolve
\item $\log_b(xy) = \log_b(x) + \log_b(y)$

\pause\transdissolve
\item $\log_b(x/y) = \log_b(x) - \log_b(y)$

\pause\transdissolve
\item $\log_b(x^n) = n\log_b(x)$

\pause\transdissolve
\item $\log_b(x) = \log_a(x) / \log_a(b)$
\end{enumerate}
\end{frame}

\begin{frame}[fragile]
\pause\transdissolve

Trigonometric functions are functions of an angle expressed in radians

\pause\transdissolve\bigskip

1 degree = $\frac{\pi}{180}$ radians

\pause\transdissolve\bigskip

Sine function $f(x) = \sin(x)$
\visible<3->{
\begin{center}
\begin{tikzpicture}
\footnotesize
\begin{scope}[->,thin,
	   node distance=0.5cm,
  	   block1/.style={rectangle,draw,align=center},
	   block2/.style={rectangle,align=center}]
\node [block2] (1) {$x$ (in radians)};
\node [block1] (2) [right=of 1] {$\sin(x)$};
\node [block2] (3) [right=of 2] {$y \in [-1, 1]$};
\path (1) edge node [above] {} (2);
\path (2) edge node [above] {} (3);
\end{scope}
\end{tikzpicture}
}
\end{center}

\pause\transdissolve\bigskip

Cosine function $f(x) = \cos(x)$
\visible<4->{
\begin{center}
\begin{tikzpicture}
\footnotesize
\begin{scope}[->,thin,
	   node distance=0.5cm,
  	   block1/.style={rectangle,draw,align=center},
	   block2/.style={rectangle,align=center}]
\node [block2] (1) {$x$ (in radians)};
\node [block1] (2) [right=of 1] {$\cos(x)$};
\node [block2] (3) [right=of 2] {$y \in [-1, 1]$};
\path (1) edge node [above] {} (2);
\path (2) edge node [above] {} (3);
\end{scope}
\end{tikzpicture}
}
\end{center}

\pause\transdissolve\bigskip

Tangent function $f(x) = \tan(x)$
\visible<4->{
\begin{center}
\begin{tikzpicture}
\footnotesize
\begin{scope}[->,thin,
	   node distance=0.5cm,
  	   block1/.style={rectangle,draw,align=center},
	   block2/.style={rectangle,align=center}]
\node [block2] (1) {$x$ (in radians)};
\node [block1] (2) [right=of 1] {$\tan(x)$};
\node [block2] (3) [right=of 2] {$y \in (-\infty, \infty)$};
\path (1) edge node [above] {} (2);
\path (2) edge node [above] {} (3);
\end{scope}
\end{tikzpicture}
}
\end{center}
\end{frame}

\begin{frame}[fragile]
\pause\transdissolve

Arcsine function $f(x) = \arcsin(x)$ (inverse of sine function)
\visible<2->{
\begin{center}
\begin{tikzpicture}
\footnotesize
\begin{scope}[->,thin,
	   node distance=0.5cm,
  	   block1/.style={rectangle,draw,align=center},
	   block2/.style={rectangle,align=center}]
\node [block2] (1) {$x \in [-1, 1]$ };
\node [block1] (2) [right=of 1] {$\arcsin(x)$};
\node [block2] (3) [right=of 2] {$y \in [-\frac{\pi}{2}, \frac{\pi}{2}]$};
\path (1) edge node [above] {} (2);
\path (2) edge node [above] {} (3);
\end{scope}
\end{tikzpicture}
}
\end{center}

\pause\transdissolve\bigskip

Arccosine function $f(x) = \arccos(x)$ (inverse of cosine function)
\visible<3->{
\begin{center}
\begin{tikzpicture}
\footnotesize
\begin{scope}[->,thin,
	   node distance=0.5cm,
  	   block1/.style={rectangle,draw,align=center},
	   block2/.style={rectangle,align=center}]
\node [block2] (1) {$x \in [-1, 1]$ };
\node [block1] (2) [right=of 1] {$\arccos(x)$};
\node [block2] (3) [right=of 2] {$y \in [0, \pi]$};
\path (1) edge node [above] {} (2);
\path (2) edge node [above] {} (3);
\end{scope}
\end{tikzpicture}
}
\end{center}

\pause\transdissolve\bigskip

Arctangent function $f(x) = \arctan(x)$ (inverse of tangent function)
\visible<4->{
\begin{center}
\begin{tikzpicture}
\footnotesize
\begin{scope}[->,thin,
	   node distance=0.5cm,
  	   block1/.style={rectangle,draw,align=center},
	   block2/.style={rectangle,align=center}]
\node [block2] (1) {$x \in (-\infty, \infty)$ };
\node [block1] (2) [right=of 1] {$\arctan(x)$};
\node [block2] (3) [right=of 2] {$y \in [-\frac{\pi}{2}, \frac{\pi}{2}]$};
\path (1) edge node [above] {} (2);
\path (2) edge node [above] {} (3);
\end{scope}
\end{tikzpicture}
}
\end{center}
\end{frame}

\begin{frame}[fragile]
\pause\transdissolve

The factorial function $n!$ ($n > 0$) is the product of all positive integers less than or equal to $n$ $$n! = n \times (n - 1) \times (n - 2) \times \cdots \times 2 \times 1$$

\pause\transdissolve\bigskip

For example, $5! = 5 \times 4 \times 3 \times 2 \times 1 = 120$

\pause\transdissolve\bigskip

The value of $0!$ is 1

\pause\transdissolve\bigskip

Since $(n - 1) \times (n - 2) \times \cdots \times 2 \times 1$ is $(n - 1)!$, the factorial function $n!$ can be expressed recursively as

\[
n! = \begin{dcases*}
n(n-1)! & if $n > 0$, and \\
1       & if $n = 0$
\end{dcases*}
\]
\end{frame}

\begin{frame}[fragile]
\pause\transdissolve

The Fibonacci function $F(n)$ computes the $n$th number in the Fibonacci sequence as the sum of the two preceeding numbers

\pause\transdissolve\bigskip

The first two numbers in the Fibonacci sequence are both 1, ie, $F(1) = F(2) = 1$

\pause\transdissolve\bigskip

Recursive definition for $F(n)$

\[
F(n) = \begin{dcases*}
F(n - 1) + F(n - 2) & if $n > 2$, and \\
1       & if $n = 1$ or $2$
\end{dcases*}
\]

\pause\transdissolve\bigskip

The beginning of the Fibonacci sequence is
$$1, 1, 2, 3, 5, 8, 13, 21, 34, 55, 89, 144, \dots$$
\end{frame}

\section{Operations on Functions}
\begin{frame}[fragile]
\pause\transdissolve

Operations on functions are similar to operations on numbers

\pause\transdissolve\bigskip

Sum of $f(x)$ and $g(x)$: $(f + g)(x) = f(x) + g(x)$

\pause\transdissolve\bigskip

Difference of $f(x)$ and $g(x)$: $(f - g)(x) = f(x) - g(x)$

\pause\transdissolve\bigskip

Product of $f(x)$ and $g(x)$: $(f \cdot g)(x) = f(x) \cdot g(x)$

\pause\transdissolve\bigskip

Quotient of $f(x)$ and $g(x)$: $(f / g)(x) = f(x) / g(x)$, provided $g(x) \neq 0$

\pause\transdissolve\bigskip

Composite of $f(x)$ and $g(x)$: $(f \circ g)(x) = f(g(x))$

\begin{center}
\begin{tikzpicture}
\footnotesize
\begin{scope}[->,thin,
	   node distance=0.5cm,
  	   block1/.style={rectangle,draw,align=center},
	   block2/.style={rectangle,align=center}]
\node [block2] (1) {$x$};
\node [block1] (2) [right=of 1] {$g(x)$};
\node [block1] (3) [right=of 2] {$f(x)$};
\node [block2] (4) [right=of 3] {$y$};
\path (1) edge node [above] {} (2);
\path (2) edge node [above] {} (3);
\path (3) edge node [above] {} (4);
\end{scope}
\end{tikzpicture}
\end{center}
\end{frame}

\begin{frame}[fragile]
\pause\transdissolve

Functions can have more than one input

\pause\transdissolve\bigskip

For example, the greatest common divisor function $\text{gcd}(x, y)$ of two integers $x$ and $y$ computes the largest positive integer that divides both

\pause\transdissolve\bigskip

$\text{gcd}(54, 24) = 6$
\end{frame}

\section{Basic Probability}
\begin{frame}[fragile]
\pause\transdissolve

Probability of an event $A$, denoted $P(A)$, is defined as $$P(A) = \frac{\text{number of ways } A \text{ can happen}}{\text{total number of outcomes}}$$

\pause\transdissolve\bigskip

For example, the probability $P(E)$ of rolling an even number with a six-sided die is $$P(E) = \frac{3}{6} = \frac{1}{2}$$

\pause\transdissolve\bigskip

The probability of an event can only be between 0 and 1 and can also be written as a percentage

\pause\transdissolve\bigskip

Complement rule: $P(A) = 1 - P(A')$, where $A'$ is the complement of event $A$, consisting of elements of the sample space that are not in $A$

\pause\transdissolve\bigskip

For example, the probability $P(O)$ of rolling an odd number with a six-sided die is $$P(O) = 1 - P(O') = 1 - P(E) = \frac{1}{2}$$
\end{frame}

\section{Basic Statistics}
\begin{frame}[fragile]
\pause\transdissolve

Consider a sample of $n$ numbers $x_1, x_2, \dots, x_n$

\pause\transdissolve\bigskip

The mean $\mu$ of the sample is $$\mu = \frac{x_1 + x_2 + \cdots + x_n}{n}$$

\pause\transdissolve\bigskip

The variance $\text{Var}$ of the sample is $$\frac{(x_1 - \mu)^2 + (x_2 - \mu)^2 + \cdots + (x_n - \mu)^2}{n}$$

\pause\transdissolve\bigskip

The standard deviation $\sigma$ of the sample is $$\sigma = \sqrt{\text{Var}}$$

\pause\transdissolve\bigskip

The median of the sample is the value separating the higher half from the lower half
\end{frame}
\end{document}
