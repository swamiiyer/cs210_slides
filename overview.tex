\documentclass[8pt,a4paper,compress]{beamer}
%\documentclass[8pt,a4paper,compress,handout]{beamer}

\usepackage{amsmath, amssymb, amsthm}
\usepackage{enumerate}
\usepackage{framed}
\usepackage{listings}
\usepackage{tikz}
\usepackage{wrapfig}

\usecolortheme{dove}
\useinnertheme{circles}
\beamertemplatenavigationsymbolsempty
\setbeamertemplate{headline}
{
  \leavevmode%
  \hbox{%
  \begin{beamercolorbox}[wd=\paperwidth,ht=6ex]{secsubsec}%
    \raggedright
    \hspace*{1.5em}%
    \normalsize
    \ifx\insertsection\empty\else
      \textbf{\insertsection\text{ }}%
      \ifx\insertsubsection\empty\else
        \textbf{$\bullet$\text{ }\insertsubsection}%
      \fi
    \fi
    \hspace*{2em}%
  \end{beamercolorbox}%
  }%
}
\setbeamerfont{frametitle}{size=\normalsize}
\setbeamertemplate{mini frames}{}
\setbeamertemplate{footline}[page number]

\definecolor{lightgray}{RGB}{240,240,240}
\definecolor{darkgreen}{RGB}{51,102,0}

\title{Overview}
\date{}

\lstset{
  backgroundcolor=\color{lightgray},
  basicstyle=\footnotesize\ttfamily,
  showstringspaces=false,
  commentstyle=\color{darkgreen},
  keywordstyle=\color{blue},
  stringstyle=\color{orange},
}

\begin{document}
\begin{frame}
\vfill
\titlepage
\end{frame}

\begin{frame}
\frametitle{Outline}
\tableofcontents
\end{frame}

\section{What are Algorithms and Data Structures?}
\begin{frame}[fragile]
\begin{flushright}
\tiny \textsc{An algorithm must be seen to be believed. - \href{http://en.wikipedia.org/wiki/Donald_Knuth}{Don Knuth}}
\end{flushright}

\pause
\textbf{Algorithms} Methods for solving problems suited for computer implementation.

\pause
\smallskip

Euclid's algorithm for computing the greatest common divisor of two nonnegative integers $p$ and $q$: 

\begin{itemize}
\item In English: If $q$ is 0, the answer is $p$. If not, divide $p$ by $q$ and take the remainder $r$. The answer is the greatest common divisor of $q$ and $r$.
\item In Java:
\begin{lstlisting}[language=Java]
public static int gcd(int p, int q) {
    if (q == 0) {
        return p;
    }
    int r = p % q;
    return gcd(q, r);
}
\end{lstlisting}
\end{itemize}

\pause
\smallskip

\textbf{Data Structures} Schemes for arranging data that leave them amenable to efficient processing by algorithms.
\end{frame}

\section{Summary of Topics}
\begin{frame}[fragile]
\pause
\textbf{Fundamentals}
\begin{itemize}
\item Our Java programming model; 
\item Data abstraction; 
\item Basic data structures (bags, queues, and stacks);
\item Methods for analyzing algorithm performance; and
\item Case study (union-find).
\end{itemize}

\pause
\smallskip
\textbf{Sorting}
\begin{itemize}
\item Insertion sort; 
\item Selection sort;
\item Shell sort; 
\item Merge sort; 
\item Quick sort; 
\item Heap sort; and
\item Priority queues, selection, and merging.
\end{itemize}
\end{frame}

\begin{frame}[fragile]
\pause
\textbf{Searching}
\begin{itemize}
\item Binary search trees; 
\item Balanced search trees; and
\item Hashing.
\end{itemize}

\pause
\smallskip
\textbf{Graphs}
\begin{itemize}
\item Depth-first search; 
\item Breadth-first search; 
\item Connectivity problems; 
\item Kruskal and Prim's algorithms for finding minimum spanning trees; and
\item Dijkstra and Bellman-Ford's algorithms for solving shortest-path problems.
\end{itemize}
\end{frame}

\begin{frame}[fragile]
\pause
\textbf{Strings}
\begin{itemize}
\item Faster algorithms for sorting and searching when keys are strings; 
\item Substring search;
\item Regular expression pattern matching; and
\item Data compression algorithms.
\end{itemize}

\pause
\smallskip
\textbf{Context}
\begin{itemize}
\item Event-driven simulation; 
\item B-trees; 
\item Suffix arrays; 
\item Maximum flow; and 
\item Search problems, reduction, NP-completeness.
\end{itemize}
\end{frame}

\end{document}
