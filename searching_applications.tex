\documentclass[8pt,a4paper,compress]{beamer}

\usepackage{/home/siyer/lib/slides}

\title{Searching Applications}
\date{}
\begin{document}
\begin{frame}
\vfill
\titlepage
\end{frame}

\begin{frame}
\frametitle{Outline}
\tableofcontents
\end{frame}

\section{Sets}
\begin{frame}[fragile]
\begin{itemize}
\item mathematical set: a collection of distinct keys
\begin{lstlisting}[language={},mathescape]
public interface Set<Key extends Comparable<Key>> extends Iterable<Key>

    public void add(Key key)         // add $key$ to the set
    public boolean contains(Key key) // is $key$ in the set?
    public void delete(Key key)      // delete $key$ from the set
    public int size()                // the number of keys in the set
\end{lstlisting}

\item exception filter
\begin{itemize}
\item read in a list of words from one file

\item print out all words from standard input that are (are not) in the list
\end{itemize}

\begin{lstlisting}[language={}]
$ more list.txt
was it the of
$ more tinyTale.txt
it was the best of times it was the worst of times
...
\end{lstlisting}

\begin{lstlisting}[language={}]
$ java WhiteFilter list.txt < tinyTale.txt 
it 
was 
the 
of 
...
\end{lstlisting}

\begin{lstlisting}[language={}]
$ java BlackFilter list.txt < tinyTale.txt
best 
times 
worst 
times
...
\end{lstlisting}
\end{itemize}
\end{frame}

\begin{frame}[fragile]
\begin{itemize}
\item exception filter: applications
\begin{center}
\begin{tabular}{cccc}
\textbf{application} & \textbf{purpose} & \textbf{key} & \textbf{in list} \\ \hline \\
spell checker & identify misspelled words & word & dictionary words \\
browser & mark visited pages & URL & visited pages \\
parental controls & block sites & URL & bad sites \\
chess & detect draw & board & positions \\
spam filter & eliminate spam  & IP address & spam addresses \\
credit cards & check for stolen cards & number & stolen cards
\end{tabular} 
\end{center}
\end{itemize}
\end{frame}

\begin{frame}[fragile]
\begin{itemize}
\item exception filter: implementations
\begin{lstlisting}[language=Java]
public class WhiteFilter {  
    public static void main(String[] args) {
        OrderedSet<String> set = new OrderedSet<String>();
        In in = new In(args[0]);
        while (!in.isEmpty()) {
            String word = in.readString();
            set.add(word);
        }
        while (!StdIn.isEmpty()) {
            String word = StdIn.readString();
            if (set.contains(word)) { StdOut.println(word); }
        }
    }
}
\end{lstlisting}

\begin{lstlisting}[language=Java]
public class BlackFilter {  
    public static void main(String[] args) {
        OrderedSet<String> set = new OrderedSet<String>();
        In in = new In(args[0]);
        while (!in.isEmpty()) {
            String word = in.readString();
            set.add(word);
        }
        while (!StdIn.isEmpty()) {
            String word = StdIn.readString();
            if (!set.contains(word)) { StdOut.println(word); }
        }
    }
}
\end{lstlisting}
\end{itemize}
\end{frame}

\section{Dictionary Clients}
\begin{frame}[fragile]
\begin{itemize}
\item dictionary lookup: given a comma-separated value (CSV) file, a key field, and a value field as command-line arguments, lookup the keys read from standard input and print the corresponding values if they are found

\item examples
\begin{itemize}
\item DNS lookup
\begin{lstlisting}[language={}]
$ more ip.csv
www.princeton.edu,128.112.128.15
espn.com,199.181.135.201
yahoo.com,66.94.234.13
msn.com,207.68.172.246
google.com,64.233.167.99
baidu.com,202.108.22.33
adobe.com,192.150.18.60
yahoo.co.jp,202.93.91.141
sina.com.cn,202.108.33.32
ebay.com,66.135.192.87
...
\end{lstlisting}
\begin{lstlisting}[language={}]
$ java LookupCSV ip.csv 0 1
adobe.com
192.150.18.60
www.princeton.edu
128.112.128.15
ebay.edu
Not found 
<ctrl-d>
\end{lstlisting}
\begin{lstlisting}[language={}]
$ java LookupCSV ip.csv 1 0
128.112.128.15
www.princeton.edu
999.999.999.99
Not found
<ctrl-d>
\end{lstlisting}
\end{itemize}
\end{itemize}
\end{frame}

\begin{frame}[fragile]
\begin{itemize}
\item examples (contd.)
\begin{itemize}
\item amino acid lookup
\begin{lstlisting}[language={}]
$ more amino.csv
TTT,Phe,F,Phenylalanine
TTC,Phe,F,Phenylalanine
TTA,Leu,L,Leucine
TTG,Leu,L,Leucine
TCT,Ser,S,Serine
TCC,Ser,S,Serine
TCA,Ser,S,Serine
TCG,Ser,S,Serine
TAT,Tyr,Y,Tyrosine
TAC,Tyr,Y,Tyrosine
TAA,Stop,Stop,Stop
TAG,Stop,Stop,Stop
TGT,Cys,C,Cysteine
TGC,Cys,C,Cysteine
TGA,Stop,Stop,Stop
TGG,Trp,W,Tryptophan
...
\end{lstlisting}
\begin{lstlisting}[language={}]
$ java LookupCSV amino.csv 0 3
TCT
Serine
TAG
Stop
TAC
Tyrosine
<ctrl-d>
\end{lstlisting}
\end{itemize}
\end{itemize}
\end{frame}

\begin{frame}[fragile]
\begin{itemize}
\item dictionary lookup: implementation
\begin{lstlisting}[language=Java]
public class LookupCSV {
    public static void main(String[] args) {
        int keyField = Integer.parseInt(args[1]);
        int valField = Integer.parseInt(args[2]);
        SeparateChainingHashST<String, String> st = 
            new SeparateChainingHashST<String, String>();
        In in = new In(args[0]);
        while (in.hasNextLine()) {
            String line = in.readLine();
            String[] tokens = line.split(",");
            String key = tokens[keyField];
            String val = tokens[valField];
            st.put(key, val);
        }
        while (!StdIn.isEmpty()) {
            String s = StdIn.readString();
            if (st.contains(s)) { StdOut.println(st.get(s)); }
            else                { StdOut.println("Not found"); }
        }
    }
}
\end{lstlisting}
\end{itemize}
\end{frame}

\section{Indexing Clients}
\begin{frame}[fragile]
\begin{itemize}
\item file index: given a list of files, create an index so that you can efficiently find all files containing a given query string

\begin{lstlisting}[language={}]
$ ls *.txt
aesop.txt magna.txt moby.txt sawyer.txt tale.txt 
\end{lstlisting}

\begin{lstlisting}[language={}]
$ java FileIndex *.txt
Indexing files
  aesop.txt 
  magna.txt 
  moby.txt 
  sawyer.txt 
  tale.txt 

freedom
  magna.txt 
  moby.txt 
  tale.txt
whale
  moby.txt
lamb
  aesop.txt
  sawyer.txt 
<ctrl-d>
\end{lstlisting}
\end{itemize}
\end{frame}

\begin{frame}[fragile]
\begin{itemize}
\item file index: implementation
\begin{lstlisting}[language=Java]
import java.io.File;

public class FileIndex { 
    public static void main(String[] args) {
        SeparateChainingHashST<String, Set<File>> st = 
            new SeparateChainingHashST<String, Set<File>>();
        StdOut.println("Indexing files");
        for (String filename : args) {
            StdOut.println("  " + filename);
            File file = new File(filename);
            In in = new In(file);
            while (!in.isEmpty()) {
                String word = in.readString();
                if (!st.contains(word)) { 
                    st.put(word, new OrderedSet<File>());
                }
                Set<File> set = st.get(word);
                set.add(file);
            }
        }
        while (!StdIn.isEmpty()) {
            String query = StdIn.readString();
            if (st.contains(query)) {
                Set<File> set = st.get(query);
                for (File file : set) {
                    StdOut.println("  " + file.getName());
                }
            }
        }
    }
}
\end{lstlisting}
\end{itemize}
\end{frame}

\begin{frame}[fragile]
\begin{itemize}
\item concordance: preprocess a text corpus to support concordance queries, ie, given a word, find all occurrences with their immediate contexts

\begin{lstlisting}[language={}]
$ java Concordance tale.txt
Finished building concordance

cities
tongues of the two *cities* that were blended in

majesty
their turnkeys and the *majesty* of the law fired
me treason against the *majesty* of the people in
of his most gracious *majesty* king george the third

princeton
<ctrl-d>
\end{lstlisting}
\end{itemize}
\end{frame}

\begin{frame}[fragile]
\begin{itemize}
\item concordance: implementation

\begin{lstlisting}[language=Java]
public class Concordance {
    public static void main(String[] args) {
        int CONTEXT = 5;
        In in = new In(args[0]);
        String[] words = in.readAllStrings();
        SeparateChainingHashST<String, OrderedSet<Integer>> st = 
            new SeparateChainingHashST<String, OrderedSet<Integer>>();
        for (int i = 0; i < words.length; i++) {
            String s = words[i];
            if (!st.contains(s)) { st.put(s, new OrderedSet<Integer>()); }
            OrderedSet<Integer> set = st.get(s);
            set.add(i);
        }
        StdOut.println("Finished building concordance");
        while (!StdIn.isEmpty()) {
            String query = StdIn.readString();
            OrderedSet<Integer> set = st.get(query);
            if (set == null) { set = new OrderedSet<Integer>(); }
            for (int k : set) {
                for (int i = Math.max(0, k - CONTEXT + 1); i < k; i++) {
                    StdOut.print(words[i] + " ");
                }
                StdOut.print("*" + words[k] + "* ");
                for (int i = k + 1; i < 
                    Math.min(k + CONTEXT, words.length); i++)
                    StdOut.print(words[i] + " ");
                StdOut.println();
            }
            StdOut.println();
        }  
    }
}
\end{lstlisting}
\end{itemize}
\end{frame}

\section{Sparse Vectors and Matrices}
\begin{frame}[fragile]
\begin{itemize}
\item sparse vector: standard representation
\begin{itemize}
\item constant time access to elements

\item space proportional to $N$
\end{itemize}

\item sparse vector: symbol table representation
\begin{itemize}
\item key = index, value = entry

\item efficient iterator

\item space proportional to number of nonzeros
\end{itemize}

\item sparse matrix: standard representation
\begin{itemize}
\item each row of matrix is an array

\item constant time access to elements

\item space proportional to $N^2$
\end{itemize}

\item sparse matrix: symbol table representation
\begin{itemize}
\item each row of matrix is a sparse vector

\item efficient access to elements

\item space proportional to number of nonzeros (plus $N$)
\end{itemize}

\end{itemize}
\end{frame}

\begin{frame}[fragile]
\begin{itemize}
\item sparse vector: implementation
\begin{lstlisting}[language=Java]
public class SparseVector {
    private int N; 
    private SeparateChainingHashST<Integer, Double> st; 

    public SparseVector(int N) {
        this.N  = N;
        this.st = new SeparateChainingHashST<Integer, Double>();
    }

    public void put(int i, double value) {
        if (i < 0 || i >= N) {
            throw new IndexOutOfBoundsException("Illegal index");
        }
        if (value == 0.0) { st.delete(i); }
        else              { st.put(i, value); }
    }

    public double get(int i) {
        if (i < 0 || i >= N) { 
            throw new IndexOutOfBoundsException("Illegal index");
        }
        if (st.contains(i)) { return st.get(i); }
        else                { return 0.0; }
    }

    public int nnz() { return st.size(); }

    public int size() { return N; }
    ...
}
\end{lstlisting}
\end{itemize}
\end{frame}

\begin{frame}[fragile]
\begin{itemize}
\item sparse vector: implementation (contd.)
\begin{lstlisting}[language=Java]
public class SparseVector {
    ...
    public double dot(SparseVector that) {
        if (this.N != that.N) {
            throw new IllegalArgumentException("Vector lengths disagree");
        }
        double sum = 0.0;
        if (this.st.size() <= that.st.size()) {
            for (int i : this.st.keys()) {
                if (that.st.contains(i)) { 
                    sum += this.get(i) * that.get(i); 
                }
            }
        }
        else  {
            for (int i : that.st.keys()) {
                if (this.st.contains(i)) {
                    sum += this.get(i) * that.get(i);
                }
            }
        }
        return sum;
    }

    public double norm() {
        SparseVector a = this;
        return Math.sqrt(a.dot(a));
    }
    ...
}
\end{lstlisting}
\end{itemize}
\end{frame}

\begin{frame}[fragile]
\begin{itemize}
\item sparse vector: implementation (contd.)
\begin{lstlisting}[language=Java]
public class SparseVector {
    ...
    public SparseVector scale(double alpha) {
        SparseVector c = new SparseVector(N);
        for (int i : this.st.keys()) { c.put(i, alpha * this.get(i)); }
        return c;
    }

    public SparseVector plus(SparseVector that) {
        if (this.N != that.N) {
            throw new IllegalArgumentException("Vector lengths disagree"); 
        }
        SparseVector c = new SparseVector(N);
        for (int i : this.st.keys()) { c.put(i, this.get(i)); }
        for (int i : that.st.keys()) { c.put(i, that.get(i) + c.get(i)); } 
        return c;
    }

    public String toString() {
        String s = "";
        for (int i : st.keys()) {
            s += "(" + i + ", " + st.get(i) + ") ";
        }
        return s;
    }
    ...
}
\end{lstlisting}
\end{itemize}
\end{frame}

\begin{frame}[fragile]
\begin{itemize}
\item sparse matrix: implementation
\begin{lstlisting}[language=Java]
public class SparseMatrix {
    private int N; 
    private SparseVector[] rows; 

    public SparseMatrix(int N) {
        this.N  = N;
        rows = new SparseVector[N];
        for (int i = 0; i < N; i++) { rows[i] = new SparseVector(N); }
    }

    public void put(int i, int j, double value) {
        if (i < 0 || i >= N || j < 0 || j >= N) { 
            throw new RuntimeException("Illegal index"); 
        }
        rows[i].put(j, value);
    }

    public double get(int i, int j) {
        if (i < 0 || i >= N || j < 0 || j >= N) { 
            throw new RuntimeException("Illegal index"); 
        }
        return rows[i].get(j);
    }

    public int nnz() { 
        int sum = 0;
        for (int i = 0; i < N; i++) { sum += rows[i].nnz(); }
        return sum;
    }
    ...
}
\end{lstlisting}
\end{itemize}
\end{frame}

\begin{frame}[fragile]
\begin{itemize}
\item sparse matrix: implementation (contd.)
\begin{lstlisting}[language=Java]
public class SparseMatrix {
    ...
    public SparseVector times(SparseVector x) {
        if (N != x.size()) {
            throw new RuntimeException("Dimensions disagree");
        }
        SparseVector b = new SparseVector(N);
        for (int i = 0; i < N; i++) { b.put(i, rows[i].dot(x)); }
        return b;
    }

    public SparseMatrix plus(SparseMatrix B) {
        SparseMatrix A = this;
        if (A.N != B.N) { 
            throw new RuntimeException("Dimensions disagree"); 
        }
        SparseMatrix C = new SparseMatrix(N);
        for (int i = 0; i < N; i++) { 
            C.rows[i] = A.rows[i].plus(B.rows[i]); 
        }
        return C;
    }

    public String toString() {
        String s = "N = " + N + ", nonzeros = " + nnz() + "\n";
        for (int i = 0; i < N; i++) {
            s += i + ": " + rows[i] + "\n";
        }
        return s;
    }
    ...
}
\end{lstlisting}
\end{itemize}
\end{frame}

\end{document}
