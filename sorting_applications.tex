\documentclass[8pt,a4paper,compress]{beamer}

\usepackage{/home/siyer/lib/slides}

\title{Sorting Applications}
\date{}

\begin{document}
\begin{frame}
\vfill
\titlepage
\end{frame}

\begin{frame}
\frametitle{Outline}
\tableofcontents
\end{frame}

\section{Sorting Various Types of Data}
\begin{frame}[fragile]
\pause

Our implementations sort arrays of \lstinline$Comparable$ objects, which allows us to use Java's callback mechanism to sort arrays of objects of any type that implements the \lstinline$Comparable$ interface

\pause
\bigskip

The sorting approach we are using is known in the classical literature as pointer sorting, so called because we process references to items and do not move the data itself

\pause
\bigskip

An array might not remain sorted if a client is allowed to change the values of keys after the sort, so it is wise to ensure that key values do not change by using immutable keys
\end{frame}

\begin{frame}[fragile]
\pause

There are many applications where we want to use different orders for the objects we are sorting, and the \lstinline$Comparator$ interface makes this possible

\begin{lstlisting}[language=Java]
package edu.princeton.cs.algs4;

import java.util.Comparator;

public class Insertion {
    public static void sort(Object[] a, Comparator c) {
        int N = a.length;
        for (int i = 1; i < N; i++) {
            for (int j = i; j > 0 && less(c, a[j], a[j - 1]); j--) {
                exch(a, j, j - 1);
            }
        }
    }
    
    private static boolean less(Comparator c, Object v, Object w) {     
        return c.compare (v, w) < 0; 
    }
}
\end{lstlisting}

\pause
\bigskip

In typical applications, items have multiple instance variables that might need to serve as sort keys
\begin{lstlisting}[language=Java]
Insertion.sort(a, new Transaction.WhenOrder());
\end{lstlisting}

\begin{lstlisting}[language=Java]
Insertion.sort(a, new Transaction.HowMuchOrder());
\end{lstlisting}
\end{frame}

\begin{frame}[fragile]
\pause

The \lstinline{Transaction} data type with alternate orderings
\begin{lstlisting}[language=Java]
package edu.princeton.cs.algs4;

import java.util.Arrays;
import java.util.Comparator;

public class Transaction implements Comparable<Transaction> {  
    private final String who;
    private final Date   when; 
    private final double amount;

   public Transaction(String who, Date when, double amount) {
        this.who    = who;
        this.when   = when;
        this.amount = amount;
    }

    public Transaction(String transaction) {
        String[] a = transaction.split("\\s+");
        who    = a[0];
        when   = new Date(a[1]);
        amount = Double.parseDouble(a[2]);
    }
    
    public String toString() {
        return String.format("%-10s %10s %8.2f", who, when, amount);
    }
\end{lstlisting}
\end{frame}

\begin{frame}[fragile]
\pause

\begin{lstlisting}[language=Java]
    public int compareTo(Transaction that) {
        if      (this.amount < that.amount) { return -1; }
        else if (this.amount > that.amount) { return +1; }
        else                                { return  0; }
    }    

    public static class WhoOrder implements Comparator<Transaction> {
        public int compare(Transaction v, Transaction w) { 
            return v.who.compareTo(w.who); 
        }        
    }
    
    public static class WhenOrder implements Comparator<Transaction> {
        public int compare(Transaction v, Transaction w) { 
            return v.when.compareTo(w.when); 
        }
    }
    
    public static class HowMuchOrder implements Comparator<Transaction> {
        public int compare(Transaction v, Transaction w) {
            if (v.amount < w.amount) { return -1; }
            if (v.amount > w.amount) { return +1; }
            return 0;
        }
    }
}
\end{lstlisting}
\end{frame}

\begin{frame}[fragile]
\pause

A sorting method is stable if it preserves the relative order of equal keys in the array

\pause
\bigskip

Example

\begin{minipage}{65pt}
\begin{lstlisting}[language={},backgroundcolor=\color{white}]
Chicago 09:00:00
Phoenix 09:00:03
Houston 09:00:13
Chicago 09:00:59
Houston 09:01:10
Chicago 09:03:13
Seattle 09:10:11
Seattle 09:10:25
Phoenix 09:14:25
Chicago 09:19:32
Chicago 09:19:46
Chicago 09:21:05
Seattle 09:22:43
Seattle 09:22:54
Chicago 09:25:52
Chicago 09:35:21
Seattle 09:36:14
Phoenix 09:37:44
\end{lstlisting}
\begin{center}
\tiny Sorted by time
\end{center}
\end{minipage}\hfill
\begin{minipage}{65pt}
\begin{lstlisting}[language={},backgroundcolor=\color{white}]
Chicago 09:25:52
Chicago 09:03:13
Chicago 09:21:05
Chicago 09:19:46
Chicago 09:19:32
Chicago 09:00:00
Chicago 09:35:21
Chicago 09:00:59
Houston 09:01:10
Houston 09:00:13
Phoenix 09:37:44
Phoenix 09:00:03
Phoenix 09:14:25
Seattle 09:10:25
Seattle 09:36:14
Seattle 09:22:43
Seattle 09:10:11
Seattle 09:22:54
\end{lstlisting}
\begin{center}
\tiny Sorted by location (unstable)
\end{center}
\end{minipage}\hfill
\begin{minipage}{65pt}
\begin{lstlisting}[language={},backgroundcolor=\color{white}]
Chicago 09:00:00
Chicago 09:00:59
Chicago 09:03:13
Chicago 09:19:32
Chicago 09:19:46
Chicago 09:21:05
Chicago 09:25:52
Chicago 09:35:21
Houston 09:00:13
Houston 09:01:10
Phoenix 09:00:03
Phoenix 09:14:25
Phoenix 09:37:44
Seattle 09:10:11
Seattle 09:10:25
Seattle 09:22:43
Seattle 09:22:54
Seattle 09:36:14
\end{lstlisting}
\begin{center}
\tiny Sorted by location (stable)
\end{center}
\end{minipage}

\pause
\bigskip

Insertion and merge sorts are stable, but selection, shell, quick, and heap sorts are not
\end{frame}

\section{Which Sorting Algorithm Should I Use?}
\begin{frame}[fragile]
\pause

Performance characteristics of sorting algorithms
\begin{center}
\begin{tabular}{ccccc}
algorithm & stable? & in-place? & running time & space \\ \hline
selection & no & yes & $N^2$ & 1 \\
insertion & yes & yes & between $N$ and $N^2$ & 1 \\
shell & no & yes & $N\log N$?, $N^{6/5}$? & 1 \\
merge & yes & no & $N\log N$ & $N$ \\
quick & no & yes & $N\log N$ & $\lg N$ \\
3-way quick & no & yes & between $N$ and $N\log N$ & $\lg N$ \\
heap & no & yes & $N\log N$ & 1
\end{tabular} 
\end{center}

\pause
\bigskip

For sorting numbers, we can develop efficient versions of our sorting programs by replacing \lstinline$Comparable$ with the primitive type name, and redefining \lstinline$less()$ or just replacing calls to \lstinline$less()$ with code like \lstinline$a[i] < a[j]$

\pause
\bigskip

Java's primary system sort method  \lstinline$java.util.Arrays.sort()$ represents a collection of methods, and uses 3-way quick sort for primitive-type methods and merge sort for reference-type methods
\end{frame}

\section{Reductions}
\begin{frame}[fragile]
\pause

We can use sorting algorithms to solve other problems --- a technique in algorithm design known as reduction 

\pause
\bigskip

Examples
\begin{itemize}
\item Duplicates: finding duplicates in a collection of keys
\item Median: finding the median (the value with the property that half the keys are no larger and half the keys are no smaller) of a collection of keys
\end{itemize}
\end{frame}

\section{A Brief Survey of Sorting Applications}
\begin{frame}[fragile]
\pause

Search for information: keeping data in sorted order makes it possible to efficiently search through it using binary search

\pause
\bigskip

Combinatorial search: a classic paradigm in artificial intelligence and in coping with intractable problems is
\begin{itemize}
\item To define: a set of configurations with well-defined moves from one configuration to the next; a priority for each move; a start configuration; and a goal configuration  

\item And then apply the well-known $A^\star$ algorithm, which uses a priority queue, to solve the problem
\end{itemize}

\pause
\bigskip

Graph algorithms: priority queues play a fundamental role in graph-search algorithms, such as Kruskal's and Dijkstra's algorithms, where we proceed from item to item along edges

\pause
\bigskip

Huffman compression: a classic data compression algorithm that makes use of a priority queue

\pause
\bigskip

String processing: algorithms that process strings, which are of critical importance in modern applications in cryptography and genomics, are often based on sorting
\end{frame}
\end{document}
