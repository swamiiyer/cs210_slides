\documentclass[8pt,a4paper,compress]{beamer}

\usepackage{/home/siyer/lib/slides}

\title{Sorting Applications}
\date{}

\begin{document}
\begin{frame}
\vfill
\titlepage
\end{frame}

\begin{frame}
\frametitle{Outline}
\tableofcontents
\end{frame}

\section{Sorting Various Types of Data}
\begin{frame}[fragile]
\begin{itemize}
\item our implementations sort arrays of \lstinline$Comparable$ objects, which allows us to use Java's callback mechanism to sort arrays of objects of any type that implements the \lstinline$Comparable$ interface

\item the approach we are using is known in the classical literature as pointer sorting, so called because we process references to items and do not move the data itself

\item another advantage of using references is that we avoid the cost of moving full items --- the cost saving is significant for arrays with large items (and small keys) because the comparison needs to access just a small part of the item, and most of the item is not even touched during the sort

\item an array might not remain sorted if a client is allowed to change the values of keys after the sort, so it is wise to ensure that key values do not change by using immutable keys
\end{itemize}
\end{frame}

\begin{frame}[fragile]
\begin{itemize}
\item there are many applications where, depending on the situation, we want to use different orders for the objects that we are sorting

\item the \lstinline$Comparator$ interface allows us to build multiple orders within a single class --- it has
a single public method \lstinline$compare()$ that compares two objects
\begin{lstlisting}[language=Java]
public class Insertion {
    ...
    public static void sort(Object[] a, Comparator c) {
        int N = a.length;
        for (int i = 1; i < N; i++) {
            for (int j = i; j > 0 && less(c, a[j], a[j - 1]); j--) {
                exch(a, j, j - 1);
            }
        }
    }
    
    private static boolean less(Comparator c, Object v, Object w) {     
        return c.compare (v, w) < 0; 
    }
    ...
}
\end{lstlisting}

\item in typical applications, items have multiple instance variables that might need to serve as sort keys
\begin{lstlisting}[language=Java]
Insertion.sort(a, new Transaction.WhenOrder());
\end{lstlisting}

\begin{lstlisting}[language=Java]
Insertion.sort(a, new Transaction.HowMuchOrder());
\end{lstlisting}
\end{itemize}
\end{frame}

\begin{frame}[fragile]
\begin{itemize}
\item alternate orderings
\begin{lstlisting}[language=Java]
import java.util.Comparator;

public class Transaction {  
    ...
    public static class WhoOrder implements Comparator<Transaction> {
        public int compare(Transaction v, Transaction w) { 
            return v.who.compareTo(w.when); 
        }        
    }
    
    public static class WhenOrder implements Comparator<Transaction> {
        public int compare(Transaction v, Transaction w) { 
            return v.when.compareTo(w.when); 
        }
    }
    
    public static class HowMuchOrder implements Comparator<Transaction> {
        public int compare(Transaction v, Transaction w) {
            if (v.amount < w.amount) { return -1; }
            if (v.amount > w.amount) { return +1; }
            return 0;
        }
    }
    ...
}
\end{lstlisting}

\item the flexibility to use comparators is also useful for priority queues

\item a sorting method is stable if it preserves the relative order of equal keys in the array --- insertion and merge sorts are stable, but selection, shell, quick, and heap sorts are not
\end{itemize}
\end{frame}

\section{Which Sorting Algorithm Should I Use?}
\begin{frame}[fragile]
\begin{itemize}
\item performance characteristics of sorting algorithms
\begin{center}
\begin{tabular}{ccccc}
\textbf{algorithm} & \textbf{stable?} & \textbf{in-place?} & \textbf{running time} & \textbf{space} \\ \hline \\
selection & no & yes & $N^2$ & 1 \\
insertion & yes & yes & between $N$ and $N^2$ & 1 \\
shell & no & yes & $N\log N$?, $N^{6/5}$? & 1 \\
merge & yes & no & $N\log N$ & $N$ \\
quick & no & yes & $N\log N$ & $\lg N$ \\
3-way quick & no & yes & between $N$ and $N\log N$ & $\lg N$ \\
heap & no & yes & $N\log N$ & 1
\end{tabular} 
\end{center}

\item in some performance-critical applications, the focus may be on sorting numbers, and in such cases, we can develop efficient versions of our sorting programs by replacing \lstinline$Comparable$ with the primitive type name, and redefining \lstinline$less()$ or just replacing calls to \lstinline$less()$ with code like \lstinline$a[i] < a[j]$

\item Java's primary system sort method  \lstinline$java.util.Arrays.sort()$ represents a collection of methods, and uses 3-way quick sort for primitive-type methods and merge sort for reference-type methods
\end{itemize}
\end{frame}

\section{Reductions}
\begin{frame}[fragile]
\begin{itemize}
\item we can use sorting algorithms to solve other problems --- a technique in algorithm design known as reduction 

\item examples
\begin{itemize}
\item duplicates: finding duplicate keys in an array of \lstinline$Comparable$ objects
\item rankings: finding the Kendall tau distance between two rankings (a ranking or permutation is an array of $N$ integers where each of the integers between 0 and $N-1$ appears exactly once), defined as the number of pairs that are in different order in the two rankings
\item priority-queue reductions: finding $M$ items in an input stream with the highest keys or merging $M$ sorted input streams together to make a sorted output stream can be reduced to operations on priority queues
\item median and order statistics: finding the median (the value with the property that half the keys are no larger and half the keys are no smaller) and other order statistics of a collection of keys
\end{itemize}
\end{itemize}
\end{frame}

\section{A Brief Survey of Sorting Applications}
\begin{frame}[fragile]
\begin{itemize}
\item commercial computing: government organizations, financial institutions, and commercial enterprises organize much of their information by sorting it

\item search for information: keeping data in sorted order makes it possible to efficiently search through it using binary search

\item the classic operations research (a field that develops and applies mathematical models for problem-solving and decision-making) problem known as scheduling makes use of sorting to produce sub-optimal schedules

\item event-driven simulation: many scientific applications involve simulation, where the point of the computation is to model some aspect of the real world in order to be able to better understand it, and such applications often make use of priority queues

\item numerical computations: scientific computing is often concerned with accuracy (how close are we to the true answer?), and some numerical algorithms use priority queues and sorting to control accuracy in calculations
\end{itemize}
\end{frame}

\begin{frame}[fragile]
\begin{itemize}
\item combinatorial search: a classic paradigm in artificial intelligence and in coping with intractable problems is
\begin{itemize}
\item to define a set of configurations with well-defined moves from one configuration to the next; a priority associated with each move; a start configuration; and a goal configuration  

\item and then apply the well-known $A^\star$ algorithm, which uses a priority queue, to solve the problem
\end{itemize}

\item graph algorithms: priority queues play a fundamental role in algorithms in graph search, such as Dijkstra's, Prim's, Kruskal's algorithms, where we proceed from item to item along edges

\item Huffman compression: a classic data compression algorithm that makes use of a priority queue

\item string processing: algorithms that process strings, which are of critical importance in modern applications in cryptography and genomics, are often based on sorting
\end{itemize}
\end{frame}
\end{document}
