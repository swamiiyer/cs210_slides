\documentclass[8pt,a4paper,compress]{beamer}

\usepackage{/home/siyer/lib/slides}

\title{Symbol Tables}
\date{}

\begin{document}
\begin{frame}
\vfill
\titlepage
\end{frame}

\begin{frame}
\frametitle{Outline}
\tableofcontents
\end{frame}

\section{What is a Symbol Table?}
\begin{frame}[fragile]
\begin{itemize}
\item a symbol table is a data structure for key-value pairs that supports two operations: insert (put) a new pair into the table and search (get) the value associated with a given key

\item applications 
\begin{center}
\begin{tabular}{cccc}
\textbf{application} & \textbf{purpose} & \textbf{key} & \textbf{value} \\ \hline \\
dictionary & find definition & word & definition \\
book index & find relevant pages & term & list of page numbers \\
file share & find song to download & name of song & computer ID \\
web search & find relevant web pages & keyword & list of page names \\
compiler & find type and value & variable name & type and value \\
\end{tabular} 
\end{center}

\item conventions
\begin{itemize}
\item no duplicate keys are allowed; when a client puts a key-value pair into a table already containing that key (and an associated value), the new value replaces the old one

\item keys must not be \lstinline{null}

\item no key can be associated with the value \lstinline{null}; otherwise the key is deleted

\item deleting a key involves removing the key (and the associated value) from the table immediately.
\end{itemize}
\end{itemize}
\end{frame}

\section{API}
\begin{frame}[fragile]
\begin{itemize}
\item basic symbol table
\begin{lstlisting}[language={},mathescape]
public interface ST<Key, Value>

    void put(Key key, Value val) // put key-value pair into the table
                                 // (remove key from table if value is null)
    Value get(Key key)           // value paired with $key$, or null
    void delete(Key key)         // remove $key$ (and its value) from table
    boolean contains(Key key)    // is there a value paired with $key$?
    boolean isEmpty()            // is the table empty?
    int size()                   // number of key-value pairs in the table
    Iterable<Key> keys()         // all the keys in the table
\end{lstlisting}
\end{itemize}
\end{frame}

\begin{frame}[fragile]
\begin{itemize}
\item ordered symbol table
\begin{lstlisting}[language={},mathescape]
public interface OrderedST<Key extends Comparable<Key>, Value>

    void put(Key key, Value val) // put key-value pair into the table
                                 // (remove key from table if value is null)
    Value get(Key key)           // value paired with $key$, or null
    void delete(Key key)         // remove $key$ (and its value) from table
    boolean contains(Key key)    // is there a value paired with $key$?
    boolean isEmpty()            // is the table empty?
    int size()                   // number of key-value pairs in the table
    Key min()                    // smallest key
    Key max()                    // largest key
    Key floor(Key key)       // largest key less than or equal to $key$
    Key ceiling(Key key)     // smallest key greater than or equal to $key$
    int rank(Key key)        // number of keys less than $key$
    Key select(int k)        // key of rank $k$
    void deleteMin()         // delete smallest key
    void deleteMax()         // delete largest key
    int size(Key lo, Key hi) // number of keys in $[lo, hi]$
    Iterable<Key> keys(Key lo, Key hi) // keys in $[lo, hi]$, in sorted order
    Iterable<Key> keys()               // all keys, in sorted order    
\end{lstlisting}
\end{itemize}
\end{frame}

\section{Sample Clients}
\begin{frame}[fragile]
\begin{itemize}
\item test client
\begin{lstlisting}[language=Java]
public class LinkedST implements ST<Key, Value> {
    ...
    public static void main(String[] args) {
        LinkedST<String, Integer> st = new LinkedST<String, Integer>();
        for (int i = 0; !StdIn.isEmpty(); i++) {
            String key = StdIn.readString();
            st.put(key, i);
        }
        for (String s : st.keys()) {
            StdOut.println(s + " " + st.get(s));
        }
    }
}
\end{lstlisting}

\begin{lstlisting}[language={}]
$ java LinkedST 
S E A R C H E X A M P L E
<ctrl-d>
L 11
P 10
M 9
X 7
H 5
C 4
R 3
A 8
E 12
S 0
\end{lstlisting}
\end{itemize}
\end{frame}

\begin{frame}[fragile]
\begin{itemize}
\item performance client
\begin{lstlisting}[language=Java]
public class FrequencyCounter {
    public static void main(String[] args) {
        int distinct = 0, words = 0;
        int minlen = Integer.parseInt(args[0]);
        ResizingArrayST<String, Integer> st = 
            new ResizingArrayST<String, Integer>();
        while (!StdIn.isEmpty()) {
            String key = StdIn.readString();
            if (key.length() < minlen) { continue; }
            words++;
            if (st.contains(key)) { st.put(key, st.get(key) + 1); }
            else {
                st.put(key, 1);
                distinct++;
            }
        }
        String max = "";
        st.put(max, 0);
        for (String word : st.keys()) {
            if (st.get(word) > st.get(max)) { max = word; }
        }
        StdOut.println(max + " " + st.get(max));
        StdOut.println("distinct = " + distinct);
        StdOut.println("words    = " + words);
    }
}
\end{lstlisting}

\begin{lstlisting}[language={}]
$ java FrequencyCounter 8 < tale.txt 
business 122
distinct = 5126
words    = 14346
\end{lstlisting}
\end{itemize}
\end{frame}

\section{Implementations}
\begin{frame}[fragile]
\begin{itemize}
\item linked list based implementation of the basic symbol table API

\begin{lstlisting}[language=Java]
public class LinkedST<Key, Value> implements ST<Key, Value>{
    private int N; 
    private Node first; 

    private class Node {
        private Key key;
        private Value val;
        private Node next;

        public Node(Key key, Value val, Node next)  {
            this.key  = key;
            this.val  = val;
            this.next = next;
        }
    }

    public int size() { return N; }

    public boolean isEmpty() { return size() == 0; }

    public boolean contains(Key key) { return get(key) != null; }

    public Value get(Key key) {
        for (Node x = first; x != null; x = x.next) {
            if (key.equals(x.key)) { return x.val; }
        }
        return null;
    }
    ...
\end{lstlisting}
\end{itemize}
\end{frame}

\begin{frame}[fragile]
\begin{itemize}
\item linked list based implementation of the basic symbol table API (contd.)
\begin{lstlisting}[language=Java]
    ...    
    public void put(Key key, Value val) {
        if (val == null) { delete(key); return; }
        for (Node x = first; x != null; x = x.next) {
            if (key.equals(x.key)) { x.val = val; return; }
        }
        first = new Node(key, val, first);
        N++;
    }

    public void delete(Key key) {
        first = delete(first, key);
    }

    private Node delete(Node x, Key key) {
        if (x == null) { return null; }
        if (key.equals(x.key)) { N--; return x.next; }
        x.next = delete(x.next, key);
        return x;
    }

    public Iterable<Key> keys()  {
        LinkedQueue<Key> queue = new LinkedQueue<Key>();
        for (Node x = first; x != null; x = x.next) {
            queue.enqueue(x.key);
        }
        return queue;
    }
    ...
}
\end{lstlisting}
\end{itemize}
\end{frame}

\begin{frame}[fragile]
\begin{itemize}
\item array based implementation of the ordered symbol table API

\begin{lstlisting}[language=Java]
import java.util.NoSuchElementException;

public class ResizingArrayST<Key extends Comparable<Key>, Value> implements 
    OrderedST<Key, Value> {
    private static final int INIT_CAPACITY = 2;
    private Key[] keys;
    private Value[] vals;
    private int N = 0;

    public ResizingArrayST() { this(INIT_CAPACITY); }   

    public ResizingArrayST(int capacity) { 
        keys = (Key[]) new Comparable[capacity]; 
        vals = (Value[]) new Object[capacity]; 
    }   

    private void resize(int capacity) {
        Key[]   tempk = (Key[]) new Comparable[capacity];
        Value[] tempv = (Value[]) new Object[capacity];
        for (int i = 0; i < N; i++) {
            tempk[i] = keys[i];
            tempv[i] = vals[i];
        }
        vals = tempv;
        keys = tempk;
    }

    public boolean contains(Key key) { return get(key) != null; }

    public int size() { return N; }

    public boolean isEmpty() { return size() == 0; }
    ...
\end{lstlisting}
\end{itemize}
\end{frame}

\begin{frame}[fragile]
\begin{itemize}
\item array based implementation of the ordered symbol table API (contd.)
\begin{lstlisting}[language=Java]
    ...        
    public Value get(Key key) {
        if (isEmpty()) { return null; }
        int i = rank(key); 
        if (i < N && keys[i].compareTo(key) == 0) { return vals[i]; }
        return null;
    } 

    public int rank(Key key) {
        int lo = 0, hi = N - 1; 
        while (lo <= hi) { 
            int m = lo + (hi - lo) / 2; 
            int cmp = key.compareTo(keys[m]); 
            if      (cmp < 0) { hi = m - 1; }
            else if (cmp > 0) { lo = m + 1; } 
            else              { return m;   }
        } 
        return lo;
    } 
    ...
\end{lstlisting}
\end{itemize}
\end{frame}

\begin{frame}[fragile]
\begin{itemize}
\item array based implementation of the ordered symbol table API (contd.)
\begin{lstlisting}[language=Java]
    ...
    public void put(Key key, Value val) {
        if (val == null) { delete(key); return; }
        int i = rank(key);
        if (i < N && keys[i].compareTo(key) == 0) {
            vals[i] = val;
            return;
        }
        if (N == keys.length) { resize(2 * keys.length); }
        for (int j = N; j > i; j--)  {
            keys[j] = keys[j - 1];
            vals[j] = vals[j - 1];
        }
        keys[i] = key;
        vals[i] = val;
        N++;
    } 

    public void delete(Key key)  {
        if (isEmpty()) { return; }
        int i = rank(key);
        if (i == N || keys[i].compareTo(key) != 0) { return; }
        for (int j = i; j < N - 1; j++)  {
            keys[j] = keys[j + 1];
            vals[j] = vals[j + 1];
        }
        N--;
        keys[N] = null; 
        vals[N] = null;
        if (N > 0 && N == keys.length / 4) { resize(keys.length / 2); }
    } 
    ...
\end{lstlisting}
\end{itemize}
\end{frame}

\begin{frame}[fragile]
\begin{itemize}
\item array based implementation of the ordered symbol table API (contd.)
\begin{lstlisting}[language=Java]
    ...    
    public void deleteMin() {
        if (isEmpty()) { 
            throw new 
                NoSuchElementException("Symbol table underflow error");
        }
        delete(min());
    }

    public void deleteMax() {
        if (isEmpty()) { 
            throw new 
                NoSuchElementException("Symbol table underflow error");
        }
        delete(max());
    }

    public Key min() {
        if (isEmpty()) { return null; }
        return keys[0]; 
    }

    public Key max() {
        if (isEmpty()) { return null; }
        return keys[N - 1];
    }

    public Key select(int k) {
        if (k < 0 || k >= N) { return null; }
        return keys[k];
    }
    ...
\end{lstlisting}
\end{itemize}
\end{frame}

\begin{frame}[fragile]
\begin{itemize}
\item array based implementation of the ordered symbol table API (contd.)
\begin{lstlisting}[language=Java]
    ...
    public Key floor(Key key) {
        int i = rank(key);
        if (i < N && key.compareTo(keys[i]) == 0) { return keys[i]; }
        if (i == 0) { return null; }
        return keys[i - 1];
    }

    public Key ceiling(Key key) {
        int i = rank(key);
        if (i == N) { return null; }
        return keys[i];
    }

    public int size(Key lo, Key hi) {
        if (lo.compareTo(hi) > 0) { return 0; }
        if (contains(hi)) { return rank(hi) - rank(lo) + 1; }
        return rank(hi) - rank(lo);
    }

    public Iterable<Key> keys() { return keys(min(), max()); }
    ...
\end{lstlisting}
\end{itemize}
\end{frame}

\begin{frame}[fragile]
\begin{itemize}
\item array based implementation of the ordered symbol table API (contd.)
\begin{lstlisting}[language=Java]
    ...
    public Iterable<Key> keys(Key lo, Key hi) {
        LinkedQueue<Key> queue = new LinkedQueue<Key>(); 
        if (lo == null && hi == null) { return queue; }
        if (lo == null) {
            throw new NullPointerException("lo is null in keys()");
        }
        if (hi == null) {
            throw new NullPointerException("hi is null in keys()");
        }
        if (lo.compareTo(hi) > 0) { return queue; }
        for (int i = rank(lo); i < rank(hi); i++) {
            queue.enqueue(keys[i]);
        }
        if (contains(hi)) {
            queue.enqueue(keys[rank(hi)]);
        }
        return queue; 
    }    
    ...
}
\end{lstlisting}
\end{itemize}
\end{frame}

\section{Performance}
\begin{frame}[fragile]
\begin{itemize}
\item when studying symbol table implementations, we count comparisons, and in (rare) cases where comparisons are not in the inner loop, we count array accesses

\item cost summary for symbol table implementations
\begin{center}
\begin{tabular}{ccc}
\textbf{operation} & \textbf{unordered linked list} & \textbf{ordered array} \\ \hline \\
search (worst case) & $N$ & $\lg N$ \\
insert (worst case) & $N$ & $2N$ \\
search hit (average case) & $N/2$ & $\lg N$ \\
insert (average case) & $N$ & $N$ \\
efficiently supports ordered operations? & no & yes
\end{tabular} 
\end{center}
\end{itemize}
\end{frame}

\end{document}
