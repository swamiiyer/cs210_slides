\documentclass[8pt,a4paper,compress]{beamer}

\usepackage{/home/siyer/lib/slides}

\title{Symbol Tables}
\date{}

\begin{document}
\begin{frame}
\vfill
\titlepage
\end{frame}

\section{What is a Symbol Table?}
\begin{frame}[fragile]
\pause

A symbol table is a data structure for key-value pairs that supports two operations: insert (put) a new pair into the table and search (get) the value associated with a given key

\pause
\bigskip

Applications 
\begin{center}
\begin{tabular}{cccc}
application & purpose & key & value \\ \hline
dictionary & find definition & word & definition \\
book index & find relevant pages & term & list of page numbers \\
file share & find song to download & name of song & computer ID \\
web search & find relevant web pages & keyword & list of page names \\
compiler & find type and value & variable name & type and value \\
\end{tabular} 
\end{center}
\end{frame}

\begin{frame}[fragile]
\pause

Conventions
\begin{itemize}
\pause
\item No duplicate keys are allowed; when a client puts a key-value pair into a table already containing that key (and an associated value), the new value replaces the old one

\pause
\item Keys must not be \lstinline{null}

\pause
\item No key can be associated with the value \lstinline{null}; otherwise the key is deleted

\pause
\item Deleting a key involves removing the key (and the associated value) from the table immediately
\end{itemize}
\end{frame}

\section{API}
\begin{frame}[fragile]
\pause

Basic symbol table
\begin{center}
\begin{tabular}{cc}
method & description \\ \hline
\lstinline$ST()$ & initialize an empty symbol table \\
\lstinline$int size()$ & number of key-value pairs in the table \\
\lstinline$boolean isEmpty()$ & is the table empty? \\
\lstinline$boolean contains(Key key)$ & is there a value paired with $key$? \\
\lstinline$Value get(Key key)$ & value paired with $key$, or \lstinline$null$ \\
\lstinline$void put(Key key, Value val)$ & \makecell{put $key$-$value$ pair into the table (remove $key$ \\ from table if $value$ is \lstinline$null$)} \\
\lstinline$void delete(Key key)$ & remove $key$ (and its value) from table \\
\lstinline$Iterable<Key> keys()$ & all the keys in the table
\end{tabular} 
\end{center}
\end{frame}

\begin{frame}[fragile]
\pause

Ordered symbol table
\begin{center}
\begin{tabular}{cc}
method & description \\ \hline
\lstinline$ST()$ & initialize an empty symbol table \\
\lstinline$int size()$ & number of key-value pairs in the table \\
\lstinline$boolean isEmpty()$ & is the table empty? \\
\lstinline$boolean contains(Key key)$ & is there a value paired with $key$? \\
\lstinline$Value get(Key key)$ & value paired with $key$, or \lstinline$null$ \\
\lstinline$int rank(Key key)$ & number of keys less than $key$ \\
\lstinline$void put(Key key, Value val)$ & \makecell{put $key$-$value$ pair into the table (remove $key$ \\ from table if value is \lstinline$null$)} \\
\lstinline$void delete(Key key)$ & remove $key$ (and its value) from table \\
\lstinline$void deleteMin()$ & delete smallest key \\
\lstinline$void deleteMax()$ & delete largest key \\
\lstinline$Key min()$ & smallest key \\
\lstinline$Key max()$ & largest key \\
\lstinline$Key select(int k)$ & key of rank $k$ \\
\lstinline$Key floor(Key key)$ & largest key less than or equal to $key$ \\
\lstinline$Key ceiling(Key key)$ & smallest key greater than or equal to $key$ \\
\lstinline$int size(Key lo, Key hi)$ & number of keys in $[lo, hi]$ \\
\lstinline$Iterable<Key> keys(Key lo, Key hi)$ & keys in $[lo, hi]$, in sorted order \\
\lstinline$Iterable<Key> keys()$ & all keys, in sorted order    
\end{tabular} 
\end{center}
\end{frame}

\section{Sample Clients}
\begin{frame}[fragile]
\pause

Test client

\smallskip

\begin{lstlisting}[language=Java,style=focusin]
package edu.princeton.cs.algs4;

public class SequentialSearchST {
    public static void main(String[] args) {
        SequentialSearchST<String, Integer> st = 
            new SequentialSearchST<String, Integer>();
        for (int i = 0; !StdIn.isEmpty(); i++) {
            String key = StdIn.readString();
            st.put(key, i);
        }
        for (String s : st.keys()) {
            StdOut.println(s + " " + st.get(s));
        }
    }
}
\end{lstlisting}

\pause
\bigskip

\begin{lstlisting}[language={},style=focusin]
$ java edu.princeton.cs.algs4.SequentialSearchST 
S E A R C H E X A M P L E
<ctrl-d>
L 11
P 10
M 9
X 7
H 5
C 4
R 3
A 8
E 12
S 0
\end{lstlisting}
\end{frame}

\begin{frame}[fragile]
\pause

Performance client

\bigskip

\begin{lstlisting}[language=Java,style=focusin]
package edu.princeton.cs.algs4;

public class FrequencyCounter {
    public static void main(String[] args) {
        int distinct = 0, words = 0;
        int minlen = Integer.parseInt(args[0]);
        BinarySearchST<String, Integer> st = 
            new BinarySearchST<String, Integer>();
        while (!StdIn.isEmpty()) {
            String key = StdIn.readString();
            if (key.length() < minlen) { continue; }
            words++;
            if (st.contains(key)) { st.put(key, st.get(key) + 1); }
            else {
                st.put(key, 1);
                distinct++;
            }
        }
        String max = "";
        st.put(max, 0);
        for (String word : st.keys()) {
            if (st.get(word) > st.get(max)) { max = word; }
        }
        StdOut.println(max + " " + st.get(max));
        StdOut.println("distinct = " + distinct);
        StdOut.println("words    = " + words);
    }
}
\end{lstlisting}

\pause
\bigskip

\begin{lstlisting}[language={},style=focusin]
$ java edu.princeton.cs.algs4.FrequencyCounter 8 < tale.txt 
business 122
distinct = 5126
words    = 14346
\end{lstlisting}
\end{frame}

\section{Implementations}
\begin{frame}[fragile]
\pause

Linked list based implementation of the basic symbol table API

\smallskip

\begin{lstlisting}[language=Java,style=focusin]
package edu.princeton.cs.algs4;

public class SequentialSearchST<Key, Value> {
    private int N; 
    private Node first; 

    private class Node {
        private Key key;
        private Value val;
        private Node next;

        public Node(Key key, Value val, Node next)  {
            this.key  = key;
            this.val  = val;
            this.next = next;
        }
    }

    public SequentialSearchST() { 
        N = 0; 
        first = null; 
    }

    public int size() { return N; }

    public boolean isEmpty() { return size() == 0; }

    public boolean contains(Key key) { return get(key) != null; }
\end{lstlisting}
\end{frame}

\begin{frame}[fragile]
\pause

\begin{lstlisting}[language=Java,style=focusin]
    public Value get(Key key) {
        for (Node x = first; x != null; x = x.next) {
            if (key.equals(x.key)) { return x.val; }
        }
        return null;
    }
    
    public void put(Key key, Value val) {
        if (val == null) { delete(key); return; }
        for (Node x = first; x != null; x = x.next) {
            if (key.equals(x.key)) { x.val = val; return; }
        }
        first = new Node(key, val, first);
        N++;
    }

    public void delete(Key key) {
        first = delete(first, key);
    }

    private Node delete(Node x, Key key) {
        if (x == null) { return null; }
        if (key.equals(x.key)) { N--; return x.next; }
        x.next = delete(x.next, key);
        return x;
    }

    public Iterable<Key> keys()  {
        Queue<Key> queue = new Queue<Key>();
        for (Node x = first; x != null; x = x.next) {
            queue.enqueue(x.key);
        }
        return queue;
    }
}
\end{lstlisting}
\end{frame}

\begin{frame}[fragile]
\pause

Array based implementation of the ordered symbol table API

\smallskip

\begin{lstlisting}[language=Java,style=focusin]
package edu.princeton.cs.algs4;

import java.util.NoSuchElementException;

public class BinarySearchST<Key extends Comparable<Key>, Value> {
    private static final int INIT_CAPACITY = 2;
    private Key[] keys;
    private Value[] vals;
    private int N;

    public BinarySearchST() { 
        keys = (Key[]) new Comparable[INIT_CAPACITY]; 
        vals = (Value[]) new Object[INIT_CAPACITY]; 
        N = 0;
    }   

    private void resize(int capacity) {
        Key[]   tempk = (Key[]) new Comparable[capacity];
        Value[] tempv = (Value[]) new Object[capacity];
        for (int i = 0; i < N; i++) {
            tempk[i] = keys[i];
            tempv[i] = vals[i];
        }
        keys = tempk;
        vals = tempv;
    }

    public boolean contains(Key key) { return get(key) != null; }

    public int size() { return N; }

    public boolean isEmpty() { return size() == 0; }
\end{lstlisting}
\end{frame}

\begin{frame}[fragile]
\pause

\begin{lstlisting}[language=Java,style=focusin]
    public Value get(Key key) {
        if (isEmpty()) { return null; }
        int i = rank(key); 
        if (i < N && keys[i].compareTo(key) == 0) { return vals[i]; }
        return null;
    } 

    public int rank(Key key) {
        int lo = 0, hi = N - 1; 
        while (lo <= hi) { 
            int m = lo + (hi - lo) / 2; 
            int cmp = key.compareTo(keys[m]); 
            if      (cmp < 0) { hi = m - 1; }
            else if (cmp > 0) { lo = m + 1; } 
            else              { return m;   }
        } 
        return lo;
    } 

    public void put(Key key, Value val) {
        if (val == null) { delete(key); return; }
        int i = rank(key);
        if (i < N && keys[i].compareTo(key) == 0) {
            vals[i] = val;
            return;
        }
        if (N == keys.length) { resize(2 * keys.length); }
        for (int j = N; j > i; j--)  {
            keys[j] = keys[j - 1];
            vals[j] = vals[j - 1];
        }
        keys[i] = key;
        vals[i] = val;
        N++;
    } 
\end{lstlisting}
\end{frame}

\begin{frame}[fragile]
\pause

\begin{lstlisting}[language=Java,style=focusin]
    public void delete(Key key)  {
        if (isEmpty()) { return; }
        int i = rank(key);
        if (i == N || keys[i].compareTo(key) != 0) { return; }
        for (int j = i; j < N - 1; j++)  {
            keys[j] = keys[j + 1];
            vals[j] = vals[j + 1];
        }
        N--;
        keys[N] = null; 
        vals[N] = null;
        if (N > 0 && N == keys.length / 4) { resize(keys.length / 2); }
    } 

    public void deleteMin() {
        if (isEmpty()) { throw new NoSuchElementException(); }
        delete(min());
    }

    public void deleteMax() {
        if (isEmpty()) { throw new NoSuchElementException(); }
        delete(max());
    }

    public Key min() {
        if (isEmpty()) { return null; }
        return keys[0]; 
    }

    public Key max() {
        if (isEmpty()) { return null; }
        return keys[N - 1];
    }
\end{lstlisting}
\end{frame}

\begin{frame}[fragile]
\pause

\begin{lstlisting}[language=Java,style=focusin]
    public Key floor(Key key) {
        int i = rank(key);
        if (i < N && key.compareTo(keys[i]) == 0) { return keys[i]; }
        if (i == 0) { return null; }
        return keys[i - 1];
    }

    public Key ceiling(Key key) {
        int i = rank(key);
        if (i == N) { return null; }
        return keys[i];
    }

    public int size(Key lo, Key hi) {
        if (lo.compareTo(hi) > 0) { return 0; }
        if (contains(hi)) { return rank(hi) - rank(lo) + 1; }
        return rank(hi) - rank(lo);
    }

    public Iterable<Key> keys() { return keys(min(), max()); }

    public Iterable<Key> keys(Key lo, Key hi) {
        Queue<Key> queue = new Queue<Key>(); 
        if (lo == null && hi == null) { return queue; }
        if (lo == null) { throw new NullPointerException(); }
        if (hi == null) { throw new NullPointerException(); }
        if (lo.compareTo(hi) > 0) { return queue; }
        for (int i = rank(lo); i < rank(hi); i++) {
            queue.enqueue(keys[i]);
        }
        if (contains(hi)) { queue.enqueue(keys[rank(hi)]); }
        return queue; 
    }    
}
\end{lstlisting}
\end{frame}

\section{Performance Characteristics}
\begin{frame}[fragile]
\pause

When studying symbol table implementations, we count comparisons, and in (rare) cases where comparisons are not in the inner loop, we count array accesses

\pause
\bigskip

Cost summary for symbol table implementations
\begin{center}
\begin{tabular}{ccc}
operation & unordered linked list & ordered array \\ \hline
search & $N$ & $\lg N$ \\
insert & $N$ & $N$ \\
efficiently supports ordered operations? & no & yes
\end{tabular} 
\end{center}
\end{frame}
\end{document}
